
% 宏包 中文支持
% \usepackage{宏包1, 宏包2}
% ctex:中文支持
% amsmath:latex数学公式支持
% graphicx:插入图片
% algorithm和algorithmic:算法排版
% listings:插入代码块

% 首行缩进
% 若LaTeX默认没有段首缩进,因此要首行缩进需要进行修改。在导言区加入如下代码(距离单位一般为pt或em,pt是绝对单位;em是相对单位,表示1个中文字符宽度;本人比较喜欢em)
% 使用indentfirst宏包
% \usepackage{indentfirst}
% 设置首行缩进距离
% \setlength{\parindent}{2em}
% 单段取消缩进,放在段首即可
% \noindent
% 全局取消缩进,在想缩进的段落再进行缩进,放在导言区
% \setlength{\parindent}{0pt}
% 放在想要缩进的段落
% \hspace*{2em}段落\\

% 导言区与正文区
% 在begin{document}和end{document}之间的就是正文区,这之前的就是导言区。

% 文档类型
% \documentclass{article}article、report、book

% 英文引号
% 在LaTeX中输入英文引号时,把左侧的引号用 ` 代替

% 空格
% \quad 	1个中文字符的宽度
% \qquad	2个中文字符的宽度
% a\ b	1/3字符宽度

% 换行
% \\
% 换段
% \par
% 新页
% \newpage
% 转义字符
% \+字符

% 可选参数[htbp]
% LaTeX插入图片、表格等元素时,第一行后面有一个可选参数[htbp],例如,\begin{figure}[htbp]。
% h(here): 当前位置;将图形放置在 正文文本中给出该图形环境的地方。如果本页所剩的页面不够, 这一参数将不起作用。
% t(top): 顶部;将图形放置在页面的顶部。
% b(bottom): 底部;将图形放置在页面的底部。
% p(page): 浮动页;将图形放置在一只允许有浮动对象的页面上
% 缺省为 [tbp]

% 标题、作者、时间
% 注意:\maketitle这一行一定要在\begin{document}的后面,否则LaTeX会判定为语法错误

% 引用、脚注
% 引用:写在\begin{quote}和\end{quote}之间。
% 脚注:在需要添加脚注的文字后添加\footnote{脚注内容}即可
% 西游记\footnote{中国古典四大名著之一}小说开头写道:
% \begin{quote}引用的内容
% {\kaishu 东胜神洲有一花果山,山顶一石,受日月精华,生出一石猴。之后因为成功闯入水帘洞,被花果山诸猴拜为“美猴王”。}
% \end{quote}

% 架构
% 标题设置:一级标题\section{},耳机标题\subsection{},三级标题\subsubsection{};
% 段落设置:在一段的最后添加\par代表一段的结束;
% 目录设置:在\begin{document}内容中添加:\tableofcontents

% 链接
% 导入宏包:\usepackage{url}
% 插入超链接:\url{www.baidu.com}

% 字体,大小,颜色
% {\字体 内容}	{\songti SuperSaiyanJcy}
% {\bf 粗体}\textbf{粗体}
% {\it 斜体}\textit{斜体}
% {\sl 斜体}\textsl{斜体}
% 大小
% {\tiny Hello}{\scriptsize Hello}{\footnotesize Hello}{\small Hello}{\normalsize Hello}{\large Hello}
% 颜色
% 需要导入宏包\usepackage{color,xcolor}
% 预先定义好的颜色: red, green, blue, white, black, yellow, gray, darkgray, lightgray, brown, cyan, lime, magenta, olive, orange, pink, purple, teal, violet.

% 定义颜色的5种方式
% \definecolor{light-gray}{gray}{0.95}    % 1.灰度
% \definecolor{orange}{rgb}{1,0.5,0}      % 2.rgb
% \definecolor{orange}{RGB}{255,127,0}    % 3.RGB
% \definecolor{orange}{HTML}{FF7F00}      % 4.HTML
% \definecolor{orange}{cmyk}{0,0.5,1,0}   % 5.cmyk
% \begin{document}
% % \pagecolor{yellow}          %设置背景色为黄色
% % 使用颜色的常用方式
% \textcolor{green}{文字内容}
% \color{orange}{文字内容}
% \textcolor[rgb]{0,1,0}{文字内容}
% \color[rgb]{1,0,0}{文字内容}
% % 使用底色
% \colorbox{red}{\color{black}文字内容}红底黑字
% \fcolorbox{red}{green}{文字内容} % 框色+背景色
% \end{document}

% 单张图片
% 需要导入宏包:\usepackage{graphicx}
% %开始插入图片
% \begin{figure}[htbp] % htbp代表图片插入位置的设置
% 	\centering %图片居中
% 	%添加图片;[]中为可选参数,可以设置图片的宽高;{}中为图片的相对位置
% 	\includegraphics[width=6cm]{image.jpg}
% 	\caption{达尔文游戏} % 图片标题
% 	\label{pic1} % 图片标签
% 	\end{figure}
	
% 数学公式
% 公式支持
% 左对齐公式(行中公式):$数学公式$
% 居中公式(独立公式):$$数学公式$$
% LaTeX要输入数学公式需要导入宏包\usepackage{amsmath};若要对公式的字体进行修改,还需要引入宏包\usepackage{amsfonts}
% 使用$,即行中公式时(左对齐),数学公式与$连接处不要有空格,否则公式不会显示。
% 使用$$,即居中公式时,数学公式与$$连接处可以有空格。即$ 数学公式 $ 不显示公式。
% 使用$$时,上方要空一行。
% =不要单独打一行,否则可能会出错。
% + - * / = ( ) | , . '等符号直接在$或$$之间输入即可识别

% 在公式末尾使用\tag{编号}来实现公式手动编号,大括号内的内容可以自定义。需要使用\usepackage{amsmath}宏包,不能写在$或$$中,会报错
% \begin{equation}
% 	a^2+b^2=c^2
% 	\tag{2}
% \end{equation}

% 在导言区使用\newtheorem{example}{Example}[section]可以自定义标题样式
% \newtheorem{example}{Example}[section] % 自定义example样式
% \begin{document}
% \maketitle
% \section{Introduction}
% \begin{example}{Test1}
% Hello world!
% \end{example}
% \begin{example}{Test2}
% Hello world!
% \end{example}
% \end{document}

% 代码块
% 使用\usepackage{listings}和\usepackage{xcolor}宏包,并使用\lstset{}进行高级设置,然后使用\begin{lstlisting}[language=xxx]和\end{lstlisting}插入代码块

% 代码块基础设置
% \lstset{
% numbers=left,                          	% 在左侧显示行号
% showstringspaces=false,        			% 不显示字符串中的空格
% frame=single,                         	% 设置代码块边框
% numberstyle=\color{darkgray},               % 设置行号格式
% backgroundcolor=\color{white},              % 设置背景颜色
% keywordstyle=\color{blue},                  % 设置关键字颜色
% commentstyle=\it\color[RGB]{0,100,0},       % 设置代码注释的格式
% stringstyle=\sl\color{red},                 % 设置字符串格式
% }

% \begin{lstlisting}[language=c]
% #include <stdio.h>
% // main function
% int main() {
%     printf("Hello World!");
%     return 0;
% }
% \end{lstlisting}


% 无自动编号的标题
% LaTeX中的标题都是自动编号的,若想使用无编号的标题,可使用带*的section代码
% \section*{References}


% 参考地址https://blog.csdn.net/NSJim/article/details/109066847

\documentclass[a4paper]{article}
\usepackage[margin=1in]{geometry}

\usepackage{amsmath}
\usepackage{color,xcolor}

% \usepackage[UTF8]{ctex}
\usepackage{xeCJK}
\usepackage{fontspec}
% \setCJKmainfont[BoldFont={SimSun}]{SimSun}  % 设置 SimSun 字体,但禁用粗体
% 设置中文字体
\setCJKmainfont{Microsoft YaHei}   % 设置中文正文字体为微软雅黑
\setCJKsansfont{Microsoft YaHei}   % 设置无衬线字体为微软雅黑
\setCJKmonofont{FangSong}   % 设置等宽字体为仿宋体
% 设置英文字体
\setmainfont{Times New Roman}  % 设置英文主字体为 Times New Roman
\usepackage{setspace}
% \onehalfspacing
\doublespacing

\usepackage{indentfirst}
\setlength{\parindent}{2em}

\title{latex测试}
\author{SuperSaiyanJcy}
% \date{2024.12.11}
\date{\today}
\begin{document}
\maketitle 	% 添加这一句才能够显示标题等信息
% \tableofcontents
\newpage
\setcounter{section}{4}
\section{测试标题}
\subsection{你好\ hello}
\sloppy
\noindent
公式测试:$q=\pm ne,\ e = 1.602*10^{-19}C$\\
vabubv

\end{document}



